% !TEX encoding = UTF-8 Unicode
\documentclass[
10pt,
aspectratio=169,
]{beamer}
\setbeamercovered{transparent=10}
\usetheme[
%  showheader,
%  red,
  purple,
%  gray,
%  graytitle,
  colorblocks,
%  noframetitlerule,
]{Verona}

\usepackage[T1]{fontenc}
\usepackage[utf8]{inputenc}
\usepackage{lipsum}
% Times New Roman
\usepackage{newtxtext}
\definecolor{col}{RGB}{4,74,21}
%%%%%%%%%%%%%%%%%%%%%%%%%%%%%%%
% Mac上使用如下命令声明隶书字体,windows也有相关方式,大家可自行修改
\providecommand{\lishu}{\CJKfamily{zhli}}
%%%%%%%%%%%%%%%%%%%%%%%%%%%%%%%
\usepackage{tikz}
\usetikzlibrary{fadings}
%
%\setbeamertemplate{sections/subsections in toc}[ball]
\usepackage{xeCJK}
\usepackage{listings}
\usepackage{caption}
\usepackage{subcaption}
% \usefonttheme{professionalfonts}
\def\mathfamilydefault{\rmdefault}
\usepackage{amsmath}
\usepackage{multirow}
\usepackage{booktabs}
\usepackage{bm}
\setbeamertemplate{section in toc}{\hspace*{1em}\inserttocsectionnumber.~\inserttocsection\par}
\setbeamertemplate{subsection in toc}{\hspace*{2em}\inserttocsectionnumber.\inserttocsubsectionnumber.~\inserttocsubsection\par}
\setbeamerfont{subsection in toc}{size=\small}



\title{Machine Learning }
%\subtitle{Conferece / Transaction}
\subtitle{XXX’2025}
\author{xxx}
\paperauthor{xxx}
\paperinstitute{xxx}
\date{2025}
% \titlegraphic[width=3cm]{sysu_logo}{}




%%%%%%%%%%%%%%%%%%%%%%%%%%%%%%%%
% ----------- 标题页 ------------
%%%%%%%%%%%%%%%%%%%%%%%%%%%%%%%%



\begin{document}

\maketitle

%%%%%%%%%%%%%%%%%%%%%%%%%%%%%%%%
% ----------- FRAME ------------
%%%%%%%%%%%%%%%%%%%%%%%%%%%%%%%%


\section{Basics}
\subsection{Blocks}
	

\begin{frame}[c]{Introduction}
	\footnotesize
\begin{exampleblock}{贡献}
	\begin{itemize}
		\item 1
		\item 2
		\item 3
		\item 4
	\end{itemize}
\end{exampleblock}

	
\end{frame}	


\begin{frame}[c]{Motivation}
	\footnotesize
	%The blocks are shown below
	\begin{block}{xxx}
		\begin{itemize}
			\item 1
			\item 2
			\item 3
		\end{itemize}
	\end{block}
	\begin{exampleblock}{xxx}
		\begin{itemize}
			\item 1
		\end{itemize}
	\end{exampleblock}
	
	\end{frame}	


\begin{frame}[c]{Background}
	\footnotesize
%The blocks are shown below
\begin{figure}
	\centering
	\includegraphics[width=0.8\linewidth]{fig/fig1.png}
	% \%caption{}
\end{figure}
\begin{block}{}
	\begin{itemize}
		\item 1
		\item 2
		\item 3
		\item 4
	\end{itemize} 
\end{block}

\end{frame}	




\subsection{Enumerate \& Overlays}

\begin{frame}[c]{Design}
	\footnotesize
\begin{figure}
	\centering
	\includegraphics[width=0.8\linewidth]{fig/fig2.png}
\end{figure}
\begin{itemize}
	\item 1
	\item 2
	\item 3
\end{itemize}

\end{frame}	

	
\begin{frame}[c]{Design}
	\footnotesize
	\begin{exampleblock}{xxx}
		\begin{itemize}
			\item 1
			\item 2
			\item 3
			\item 4
			\item 5
			\item 6
			\item 7
		\end{itemize}
	\end{exampleblock}

\end{frame}

\begin{frame}[c]{Design}
	\footnotesize
	\begin{exampleblock}{xxx}
		\begin{itemize}
			\item 1
			\item 2
			\item 3
			\item 4
			\item 5
			\item 6

		\end{itemize}
	\end{exampleblock}
\end{frame}

\begin{frame}[c]{Experimental}
	\footnotesize
	\begin{block}{RQ1}
		\begin{itemize}
			\item 1
		\end{itemize}
	\end{block}
	\begin{exampleblock}{RQ2}
		\begin{itemize}
			\item 2
		\end{itemize}
	\end{exampleblock}
	\begin{alertblock}{RQ3}
		\begin{itemize}
			\item 3
		\end{itemize}
	\end{alertblock}
	\begin{exampleblock}{RQ4}
		\begin{itemize}
			\item 4
		\end{itemize}
	\end{exampleblock}
	\begin{block}{RQ5}
		\begin{itemize}
			\item 5
		\end{itemize}
	\end{block}
	
\end{frame}
\begin{frame}[c]{Experimental}
	\footnotesize
	\begin{figure}
		\centering
		%\includegraphics[height=0.7\textheight,width=0.8\linewidth]{fig/fig3.png}
		\includegraphics[height=0.7\textheight,keepaspectratio]{fig/fig3.png}
		%caption{}
	\end{figure}
\end{frame}

% \begin{frame}{横向并排两张图片}
%     \begin{columns}[onlytextwidth] % [onlytextwidth] 确保不超出页面宽度
%         \begin{column}{0.48\textwidth} % 第一列宽度(留白间距)
%             \centering
%             \includegraphics[width=\linewidth]{example-image-a} % 图片A
%             %\captionof{figure}{图片A的标题} % 可选标题
%         \end{column}
%         \hfill % 增加两列之间的间距
%         \begin{column}{0.48\textwidth} % 第二列宽度
%             \centering
%             \includegraphics[width=\linewidth]{example-image-b} % 图片B
%             %\captionof{figure}{图片B的标题} % 可选标题
%         \end{column}
%     \end{columns}
% \end{frame}


\begin{frame}{田字形式放置四张图片}
	\footnotesize
    \begin{columns}[onlytextwidth,T] % T表示顶部对齐
        % 第一列(左半边)
        \begin{column}{0.48\textwidth}
            \centering
            \includegraphics[width=\linewidth, height=0.4\textheight,keepaspectratio]{fig/fig4.png}
            \vspace{0.5em} % 图片间距
            \includegraphics[width=\linewidth, height=0.4\textheight,keepaspectratio]{fig/fig5.png}
        \end{column}
        % 第二列(右半边)
        \begin{column}{0.48\textwidth}
            \centering
            \includegraphics[width=\linewidth, height=0.4\textheight,keepaspectratio]{fig/fig6.png}
            \vspace{0.5em}
            \includegraphics[width=\linewidth, height=0.4\textheight,keepaspectratio]{fig/fig7.png}
        \end{column}
    \end{columns}
\end{frame}

\begin{frame}[c]{Disscussion}
	\footnotesize
	\begin{exampleblock}{x x x}
		\begin{itemize}
			\item 1
			\item 2
		\end{itemize}
	\end{exampleblock}
	
	
\end{frame}


	




% Thank you page
\beamertemplateshadingbackground{structure.fg!90}{structure.fg}
\begin{frame}[plain]
	\vfill
	\centering
	{
		\centering \Huge \color{white} Thanks!\\[10pt]Questions?
	}
	\vfill
\end{frame}

\end{document}


